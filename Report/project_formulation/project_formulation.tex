\documentclass{article}

% Packages
\usepackage{fancyhdr}
\usepackage{graphicx}
\usepackage{amsmath}
\usepackage{biblatex}

\addbibresource{bib.bib}

% Configuring Fancy hdr
\pagestyle{fancy}

\fancyhead[R]{Project Formulation}
\fancyhead[L]{Philip Oliver Mejer Jørgensen}

% Configuring title page
    \title{Detection of Vinyl Chloride in environmental water samples}
\author{Philip Oliver Mejer Jørgensen}
\date{\today}

% Main document
\begin{document}
\maketitle

\section*{Background}
Humans produce waste, whether that is as an individual or on an industrial level. A lot of the waste that is being produced, end up in our water supply contaminating it.
An example of a toxic waste product that can end up in the water supply is vinyl chloride \textit{(VC)}, one of the primary areas VC is found is in the production of PVC.
PVC plastic is used as pipes for plumbing, bottles, and more\cite{pvc_applications}.
Vinyl chloride is a highly volatile compound which makes it both hard to detect, and making it more dangerous.
This has created a need for a fast and simple solution for detecting the concentration of vinyl chloride in water samples.

This has led the company Water Care Guard to develop an on-site lab kit that fits in a suitcase, which is able to test a water sample for the concentration of vinyl chloride amongst other substances.

This project is done in collaboration with Roana from SDU Nano Syd and Water Care Guard.

\vspace{15mm}

\section*{Problem}


\vspace{15mm}
\section*{Timetable and milestones}
The project is divided into 3 main parts:\\
\begin{itemize}
    \item Building the database
    \item Analyzing the data
    \item User interface
\end{itemize}

\vspace{15mm}
\section*{Risk assessment}

\end{document}